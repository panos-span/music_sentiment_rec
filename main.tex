\documentclass[a4paper,12pt]{article}
\usepackage[utf8]{inputenc}
\usepackage[LGR,T1]{fontenc}
\usepackage{alphabeta}
\usepackage{amsmath}
\usepackage{float}
\usepackage{multirow}
\usepackage{amsfonts}
\usepackage{amssymb}
\usepackage{graphicx}
\usepackage{listings}
\usepackage{booktabs}  % For better looking tables
\usepackage{hyperref}
\usepackage{xcolor}
\usepackage{breakurl}
\usepackage[numbers]{natbib}

\lstset{
  basicstyle=\ttfamily\footnotesize,
  keywordstyle=\color{blue},
  commentstyle=\color{gray},
  stringstyle=\color{red},
  showstringspaces=false,
  breaklines=true,
  frame=single,
  language=R,
  extendedchars=true,
  literate={β}{{\beta}}1
}

\title{Αναγνώριση Προτύπων \\ 1η Εργαστηριακή Άσκηση \\ Χειμερινό Εξάμηνο 2024-2025 \\ Ε.ΔE.ΜΜ}
\author{Σπανάκης Παναγιώτης-Αλέξιος (ΑΜ: 03400274)}
\date{08/11/2024}

\begin{document}

\maketitle

\section*{Βήμα 1: Εξοικείωση με φασματογραφήματα στην κλίμακα mel}

\subsection*{α, β, γ) Ανάλυση των φασματογραφημάτων Mel}
Εξετάζοντας τα δύο παραδείγματα (Rock - Αρχείο 104085 και Electronic - Αρχείο 1334), παρατηρούμε τα εξής διακριτά χαρακτηριστικά:

\textbf{Είδος Rock (Αρχείο 104085):}
\begin{itemize}
    \item Υψηλότερο ενεργειακό περιεχόμενο σε όλο το εύρος συχνοτήτων (0-10000 Hz)
    \item Έντονη παρουσία στις μεσαίες συχνότητες (2000-6000 Hz)
    \item Σαφή κατακόρυφα μοτίβα που υποδεικνύουν τακτικά ρυθμικά στοιχεία
    \item Σταθερή ενέργεια στις χαμηλές συχνότητες που δείχνει έντονη παρουσία μπάσου/ντραμς
    \item Πιο ομοιόμορφη κατανομή ενέργειας που υποδηλώνει χρήση παραδοσιακών οργάνων
\end{itemize}

\textbf{Είδος Electronic (Αρχείο 1334):}
\begin{itemize}
    \item Συγκέντρωση ενέργειας στις χαμηλές συχνότητες (0-4000 Hz)
    \item Λιγότερο ομοιόμορφη φασματική κατανομή
    \item Πιο σποραδικά μοτίβα στις υψηλές συχνότητες
    \item Μεταβλητά χρονικά μοτίβα τυπικά των συνθετικών ήχων
    \item Λιγότερο καθορισμένες κατακόρυφες γραμμές, υποδεικνύοντας διαφορετικές ρυθμικές δομές
\end{itemize}

\subsection*{δ) Η Κλίμακα Mel}
Η κλίμακα Mel είναι μια αντιληπτική κλίμακα συχνοτήτων που αναπτύχθηκε για να ταιριάζει με τον τρόπο που οι άνθρωποι αντιλαμβάνονται τις διαφορές τόνου.

\textbf{1. Ορισμός και Δημιουργία:}
\begin{itemize}
    \item Δημιουργήθηκε μέσω πειραμάτων όπου οι ακροατές αναγνώριζαν ζεύγη ήχων που αντιλαμβάνονταν ως ισαπέχοντα σε τόνο
    \item Τύπος: $mel = 2595 \cdot \log_{10}(1 + \frac{f}{700})$, όπου f η συχνότητα σε Hz
    \item Περίπου γραμμική κάτω από 1000 Hz και λογαριθμική πάνω από αυτή
\end{itemize}

\textbf{2. Χρήση στην Επεξεργασία Μουσικής:}
\begin{itemize}
    \item Καλύτερη αναπαράσταση της ανθρώπινης ακουστικής αντίληψης
    \item Παρέχει μεγαλύτερη ανάλυση στις χαμηλές συχνότητες όπου η ανθρώπινη ακοή είναι πιο ευαίσθητη
    \item Μειώνει την υπολογιστική πολυπλοκότητα διατηρώντας τις αντιληπτικά σημαντικές πληροφορίες
    \item Βελτιώνει την απόδοση στην ταξινόμηση και ανάλυση μουσικής
\end{itemize}

\section*{Βήμα 2: Συγχρονισμός φασματογραφημάτων στο ρυθμό της μουσικής}

\subsection*{α) Ανάλυση Διαστάσεων}

\textbf{Διαστάσεις Φασματογραφημάτων:}
\begin{itemize}
   \item Τα αρχικά φασματογραφήματα έχουν περίπου 1291-1293 χρονικά βήματα
   \item Κάθε χρονικό βήμα περιέχει 140 χαρακτηριστικά (128 για mel + 12 για chroma)
\end{itemize}

\textbf{Αποδοτικότητα LSTM:}
\begin{itemize}
   \item Η εκπαίδευση LSTM με τόσα πολλά χρονικά βήματα δεν είναι αποδοτική για τους εξής λόγους:
   \begin{enumerate}
       \item \textbf{Υπολογιστική Πολυπλοκότητα:}
       \begin{itemize}
           \item Τα LSTM πρέπει να επεξεργαστούν κάθε χρονικό βήμα διαδοχικά
           \item Μεγάλες ακολουθίες αυξάνουν σημαντικά τον χρόνο εκπαίδευσης
           \item Απαιτούν περισσότερη μνήμη για την αποθήκευση των ενδιάμεσων καταστάσεων
       \end{itemize}
       
       \item \textbf{Πρόβλημα Εξαφανιζόμενης Κλίσης:}
       \begin{itemize}
           \item Μεγάλες ακολουθίες επιδεινώνουν το πρόβλημα των εξαφανιζόμενων κλίσεων
           \item Δυσκολεύουν την εκμάθηση μακροπρόθεσμων εξαρτήσεων
       \end{itemize}
       
       \item \textbf{Περιττή Πληροφορία:}
       \begin{itemize}
           \item Πολλά διαδοχικά χρονικά βήματα μπορεί να περιέχουν παρόμοια πληροφορία
           \item Δεν είναι απαραίτητη τόσο λεπτομερής χρονική ανάλυση για την ταξινόμηση είδους
       \end{itemize}
   \end{enumerate}
\end{itemize}

\subsection*{β) Ανάλυση Beat-Synchronized Φασματογραφημάτων}

\textbf{Σύγκριση με τα Αρχικά Φασματογραφήματα:}
\begin{itemize}
   \item \textbf{Πλεονεκτήματα:}
   \begin{enumerate}
       \item \textbf{Μειωμένη Διάσταση:}
       \begin{itemize}
           \item Σημαντικά λιγότερα χρονικά βήματα
           \item Πιο αποδοτική επεξεργασία και εκπαίδευση
       \end{itemize}
       
       \item \textbf{Μουσική Σημασία:}
       \begin{itemize}
           \item Ευθυγράμμιση με τη ρυθμική δομή της μουσικής
           \item Καλύτερη αναπαράσταση των μουσικών γεγονότων
       \end{itemize}
       
       \item \textbf{Ποιότητα Χαρακτηριστικών:}
       \begin{itemize}
           \item Διατήρηση των σημαντικών φασματικών χαρακτηριστικών
           \item Μείωση του θορύβου μέσω της διαμέσου
       \end{itemize}
   \end{enumerate}
   
   \item \textbf{Διαφορές στην Οπτικοποίηση:}
   \begin{itemize}
       \item Πιο ευδιάκριτα ρυθμικά μοτίβα
       \item Καλύτερη αναπαράσταση της μουσικής δομής
       \item Ομαλότερη χρονική εξέλιξη των χαρακτηριστικών
   \end{itemize}
\end{itemize}

\textbf{Συμπέρασμα:}
Ο συγχρονισμός με το ρυθμό προσφέρει μια πιο συμπαγή και μουσικά σημαντική αναπαράσταση, κατάλληλη για εκπαίδευση νευρωνικών δικτύων, διατηρώντας παράλληλα τα ουσιώδη χαρακτηριστικά για την ταξινόμηση του μουσικού είδους.

\section*{Βήμα 3: Εξοικείωση με χρωμογραφήματα}

Αναλύοντας τα χρωμογραφήματα για τα ίδια δύο αρχεία παρατηρούμε διαφορετικά αρμονικά χαρακτηριστικά:

\textbf{Είδος Rock (Αρχείο 104085):}
\begin{itemize}
    \item Σαφής αρμονική δομή με καθορισμένα μοτίβα νοτών
    \item Έντονη παρουσία σε συγκεκριμένες τάξεις τόνου (κυρίως E)
    \item Τακτικά μοτίβα που υποδεικνύουν σταθερές ακολουθίες συγχορδιών
    \item Καλά καθορισμένες μεταβάσεις μεταξύ διαφορετικού αρμονικού περιεχομένου
\end{itemize}

\textbf{Είδος Electronic (Αρχείο 1334):}
\begin{itemize}
    \item Πιο διάχυτη κατανομή ενέργειας στις τάξεις τόνου
    \item Έντονη δραστηριότητα στην περιοχή της νότας A
    \item Λιγότερο καθορισμένες αρμονικές ακολουθίες
    \item Θολές μεταβάσεις μεταξύ νοτών
\end{itemize}

\textbf{Βασικές Παρατηρήσεις για τα Χρωμογραφήματα:}
\begin{enumerate}
    \item \textbf{Αναπαράσταση:}
    \begin{itemize}
        \item Δείχνει την κατανομή ενέργειας στις 12 τάξεις τόνου (C έως B)
        \item Συμπτύσσει τις πληροφορίες οκτάβας για εστίαση στο αρμονικό περιεχόμενο
        \item Η φωτεινότητα υποδεικνύει την ένταση κάθε τάξης τόνου
    \end{itemize}
    
    \item \textbf{Οφέλη Μουσικής Ανάλυσης:}
    \begin{itemize}
        \item Αποκαλύπτει την αρμονική δομή ανεξάρτητα από την οκτάβα
        \item Ανθεκτικό στις αλλαγές ηχοχρώματος διατηρώντας τις αρμονικές πληροφορίες
        \item Χρήσιμο για την ανάλυση ακολουθιών συγχορδιών
        \item Βοηθά στην αναγνώριση τονικότητας και τροπικού περιεχομένου
    \end{itemize}
\end{enumerate}

Οι εκδόσεις συγχρονισμένες με το ρυθμό (beat-synchronized) ευθυγραμμίζουν αυτά τα χαρακτηριστικά με το μουσικό παλμό, καθιστώντας τα χρονικά μοτίβα πιο εμφανή και ευκολότερα στην ανάλυση.


\section*{Βήμα 4: Φόρτωση και ανάλυση δεδομένων}

\subsection*{α) Ανάλυση Κώδικα PyTorch Dataset}

Ο κώδικας υλοποιεί τις εξής βασικές λειτουργίες:

\begin{enumerate}
   \item \textbf{Κλάση SpectrogramDataset:}
   \begin{itemize}
       \item Κληρονομεί από την torch.utils.data.Dataset
       \item Διαχειρίζεται τη φόρτωση και προεπεξεργασία των φασματογραφημάτων
       \item Υποστηρίζει τόσο mel όσο και chroma χαρακτηριστικά
   \end{itemize}

   \item \textbf{Βασικές Λειτουργίες:}
   \begin{itemize}
       \item \texttt{read\_spectrogram}: Φορτώνει και διαχωρίζει τα mel και chroma χαρακτηριστικά
       \item \texttt{get\_files\_labels}: Διαβάζει τα ονόματα αρχείων και τις ετικέτες
       \item \texttt{PaddingTransform}: Εξασφαλίζει ομοιόμορφο μήκος για όλα τα δείγματα
       \item \texttt{LabelTransformer}: Μετατρέπει τις ετικέτες σε αριθμητική μορφή
   \end{itemize}

   \item \textbf{Επεξεργασία Δειγμάτων:}
   \begin{itemize}
       \item Μετατροπή σε numpy arrays τύπου float32
       \item Προσθήκη padding όπου χρειάζεται
       \item Κωδικοποίηση ετικετών
   \end{itemize}
\end{enumerate}

\subsection*{β) Συγχώνευση Κλάσεων}

Το CLASS\_MAPPING υλοποιεί:
\begin{itemize}
   \item Συγχώνευση παρόμοιων μουσικών ειδών:
   \begin{itemize}
       \item "Psych-Rock", "Post-Rock" $\rightarrow$ "Rock"
       \item "Punk" $\rightarrow$ "Metal"
       \item "Chiptune" $\rightarrow$ "Electronic"
   \end{itemize}
   \item Αφαίρεση προβληματικών κλάσεων (None):
   \begin{itemize}
       \item "Indie-Rock", "Post-Punk"
       \item "Soundtrack", "International"
       \item "Old-Time"
   \end{itemize}
\end{itemize}

\subsection*{γ) Ανάλυση Κατανομής Κλάσεων}

\textbf{1. Αρχική Κατανομή (Initial Distribution):}
\begin{itemize}
   \item 20 διαφορετικές κλάσεις (classes)
   \item Μέσος αριθμός δειγμάτων ανά κλάση: ~160 samples
   \item Εύρος: 90-160 samples ανά κλάση
   \item Παρουσία class imbalance με μερικές υποεκπροσωπούμενες κλάσεις
\end{itemize}

\textbf{2. Κατανομή Μετά το Class Mapping:}
\begin{itemize}
   \item Μείωση σε 10 βασικές κλάσεις
   \item Νέα κατανομή δειγμάτων:
   \begin{itemize}
       \item Rock: ~320 samples (συγχώνευση Rock, Psych-Rock, Post-Rock)
       \item Folk: ~320 samples (συγχώνευση Folk, Psych-Folk)
       \item Metal: ~310 samples (συγχώνευση Metal, Punk)
       \item Electronic: ~410 samples (συγχώνευση Electronic, Chiptune)
       \item Pop: ~140 samples
       \item Άλλες κλάσεις (Trip-Hop, Blues, Jazz, κλπ.): 150-160 samples
   \end{itemize}
\end{itemize}

\textbf{3. Επιπτώσεις του Class Mapping:}
\begin{enumerate}
   \item \textbf{Dataset Balance:}
   \begin{itemize}
       \item Βελτιωμένη αντιπροσώπευση των κύριων μουσικών genres
       \item Ελαφρώς αυξημένο class imbalance μεταξύ των μεγάλων (Rock, Electronic) και μικρών κλάσεων
       \item Πιο στιβαρό dataset για training
   \end{itemize}
   
   \item \textbf{Training Implications:}
   \begin{itemize}
       \item Καλύτερη generalization λόγω περισσότερων samples ανά κλάση
       \item Πιθανή ανάγκη για class weights στο training
       \item Reduced overfitting risk για τις συγχωνευμένες κλάσεις
   \end{itemize}
   
   \item \textbf{Model Performance:}
   \begin{itemize}
       \item Αναμενόμενη βελτίωση στο classification των κύριων genres
       \item Πιθανή μείωση confusion μεταξύ παρόμοιων υποκατηγοριών
       \item Καλύτερη robustness στο testing
   \end{itemize}
\end{enumerate}

\textbf{4. Προτεινόμενες Στρατηγικές:}
\begin{itemize}
   \item Χρήση class weights στο loss function για αντιμετώπιση του imbalance
   \item Εφαρμογή data augmentation για τις μικρότερες κλάσεις
   \item Monitoring του validation score ανά κλάση
   \item Cross-validation για αξιολόγηση της robustness του μοντέλου
\end{itemize}

\bibliographystyle{plainnat}
\bibliography{references}

\end{document}